%% -*- Mode: LaTeX -*-
%%
%% process-logs.tex
%% Created Fri Sep 20 14:20:08 AKDT 2019
%% Copyright (C) 2019 by Raymond E. Marcil <marcilr@gmail.com>
%%

%% ========================== Process Logs ==============================
%% ========================== Process Logs ==============================
\newpage
\section{Process Logs}
The awstats logs are processed to update the awstats databases
under \textbf{DirData}:\\
\indent\texttt{DirData="/net/dnr-atwfs1/vol/vol1/weblogs/reports/data"}\\
\\
The 2018 logs to be processed are available on the network path:\\
\indent\texttt{/net/dnr-atwfs1/vol/vol1/weblogs/YYYY/MM/DD}\\
\\
The internet access are within log files like:\\
\indent\texttt{/net/dnr-atwfs1/vol/vol1/weblogs/2018/01/01/nginx/access\_log}\\
\\
The log files need to be processed on a monthly basis for all of
2018. This can be broken down by month to:
\\
%%
%% You can add space by inserting @{\hskip whatever} between
%% the column specifiers, as in
%%
%% \begin{tabular}{l@{\hskip 1in}c@{\hskip 0.5in}c}
%%   One&Two& Three\\
%%   Four& Five& Six
%% \end{tabular}
%%
%% Adding space between columns in a table
%% https://tex.stackexchange.com/questions/16519/adding-space-between-columns-in-a-table
%%
%% How to center the table in Latex
%% https://tex.stackexchange.com/questions/162462/how-to-center-the-table-in-latex
%%
%% LATEX Table Tricks
%% Adrian P. Robson
%% adrian.robson@northumbria.ac.uk
%% 24th June, 2009
%% http://www.tex.uniyar.ac.ru/doc/tableTricks.pdf
%%
\begin{center}
\begin{tabular}{l@{\hskip0.5in}l@{\hskip1.5in}l@{\hskip0.5in}l}
\bf{Month} & \bf{Days} & \bf{Month} & \bf{Days} \\
January    & 1-31      & July       & 1-31 \\
February   & 1-28      & August     & 1-31 \\
March      & 1-31      & September  & 1-30 \\
April      & 1-30      & October    & 1-31 \\
May        & 1-31      & November   & 1-30 \\
June       & 1-30      & December   & 1-31 \\
\end{tabular}
\end{center}

%%\noindent\begin{tabular}{ l l l }
%% cell1 & cell2 & cell3 \\ 
%% cell4 & cell5 & cell6 \\  
%% cell7 & cell8 & cell9    
%%\end{tabular}

A script can then be written to loop over all the days 
of the month to process the individual days.
